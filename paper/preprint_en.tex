\documentclass[12pt, a4paper]{amsart}

% === Packages ===
\usepackage[utf8]{inputenc}
\usepackage[T1]{fontenc}
\usepackage[english]{babel}
\usepackage{amsmath, amssymb, amsthm, mathtools}
\usepackage{enumitem}
\usepackage{hyperref}
\usepackage{booktabs}
\usepackage{array}
\usepackage{microtype}

% === Theorems ===
\theoremstyle{plain}
\newtheorem{theorem}{Theorem}[section]
\newtheorem{proposition}[theorem]{Proposition}
\newtheorem{lemma}[theorem]{Lemma}
\newtheorem{corollary}[theorem]{Corollary}

\theoremstyle{definition}
\newtheorem{definition}[theorem]{Definition}
\newtheorem{notation}[theorem]{Notation}
\newtheorem{example}[theorem]{Example}
\newtheorem{conjecture}[theorem]{Conjecture}

\theoremstyle{remark}
\newtheorem{remark}[theorem]{Remark}

% === Commands ===
\newcommand{\N}{\mathbb{N}}
\newcommand{\Z}{\mathbb{Z}}
\newcommand{\Q}{\mathbb{Q}}
\newcommand{\Fp}{\mathbb{F}_p}
\newcommand{\Comp}{\mathrm{Comp}}
\newcommand{\Ev}{\mathrm{Ev}}
\newcommand{\corrSum}{\mathrm{corrSum}}
\newcommand{\ord}{\mathrm{ord}}
\DeclareMathOperator{\sgn}{sgn}

\title[Entropic barriers and nonsurjectivity]{%
  Entropic Barriers and Nonsurjectivity \\
  in the $3x+1$ Problem: the Junction Theorem}

\author{Eric Merle}
\address{Independent researcher}
\email{eric.merle@proton.me}
\date{March 2026}

\subjclass[2020]{11B83 (primary), 37A45, 94A17 (secondary)}

\keywords{Collatz problem, nonsurjectivity, Shannon entropy,
  Steiner equation, exponential sums, characters}

\begin{document}

\begin{abstract}
  We study the nonexistence of nontrivial positive cycles in
  the Collatz ($3x+1$) dynamics.
  By revisiting Steiner's equation~(1977) through the lens of
  information theory, we identify a universal entropic deficit
  $\gamma = 1 - h(1/\log_2 3) \approx 0.0500$,
  where $h$ denotes the binary Shannon entropy.
  This deficit implies that the modular evaluation map $\Ev_d$
  cannot be surjective for any cycle candidate of length
  $k \geq 18$ (\textbf{unconditional}, Theorem~1).
  Combined with the computational bound of Simons and
  de~Weger~(2005) for $k \leq 68$, we obtain a
  \textbf{Junction Theorem} (Theorem~2): for every $k \geq 2$,
  at least one obstruction---computational or
  entropic---applies.

  The residual question---excluding the specific residue~$0$
  from the image of~$\Ev_d$---is formulated as
  \textbf{Hypothesis~(H)} and analyzed via additive and
  multiplicative character sums.
  We establish unconditional bounds (Parseval cost, peeling
  lemma, Mellin--Fourier bridge) and formulate
  \textbf{Conjecture~M} as the single remaining obstacle.
  The core results are formalized in Lean~4 ($73$~theorems,
  zero \texttt{sorry}, zero axioms); a research-level
  skeleton with Mathlib adds ${\sim}\,58$ further proofs.
  Source code and a detailed research log are publicly
  available.
\end{abstract}

\maketitle

\tableofcontents

% ============================================================
\section{Introduction}\label{sec:intro}
% ============================================================

\subsection{The $3x+1$ problem}

Let $T \colon \N^* \to \N^*$ be defined by
\begin{equation}\label{eq:collatz}
  T(n) = \begin{cases}
    n/2       & \text{if } n \text{ is even,} \\
    (3n+1)/2  & \text{if } n \text{ is odd.}
  \end{cases}
\end{equation}
The \emph{Collatz conjecture}~\cite{Lagarias1985} asserts that for
every $n \geq 1$, the orbit $n, T(n), T^2(n), \ldots$ reaches~$1$.
In particular, no nontrivial positive cycle should exist.

The problem has been studied from many angles;
see~\cite{Lagarias2010} for a comprehensive survey.
Steiner~\cite{Steiner1977} and
Crandall~\cite{Crandall1978} reduced the cycle question to
a modular equation
(see also B\"ohm and Sontacchi~\cite{BoehmSontacchi1978}).
Eliahou~\cite{Eliahou1993} established lower bounds on the
minimal length of such cycles,
and Korec~\cite{Korec1994} proved that cycles of length~$k$
must satisfy $k > 17$ unless they have a special structure.
Simons and de~Weger~\cite{SimonsDeWeger2005}, using bounds
for linear forms in logarithms
(Laurent, Mignotte and Nesterenko~\cite{LMN1995}),
computationally excluded all cycles with $k \leq 68$.
Wirsching~\cite{Wirsching1998} developed a dynamical systems
perspective, while
Kontorovich and Lagarias~\cite{KontorovichLagarias2010}
introduced stochastic models.
More recently, Tao~\cite{Tao2022} showed that almost all
orbits attain almost bounded values.
Computational verification has been extended by
Barina~\cite{Barina2021} up to $5 \times 2^{60}$.

\subsection{Summary of results}

This work establishes the following results:

\begin{enumerate}[label=(\roman*)]
  \item \textbf{Theorem~\ref{thm:nonsurj}} (unconditional).
    For every $k \geq 18$ with $d(k) > 0$,
    the evaluation map $\Ev_d$ is not surjective.
  \item \textbf{Theorem~\ref{thm:junction}} (unconditional
    for the disjunction).
    For every $k \geq 2$, at least one obstruction
    (computational or entropic) applies.
    The complete exclusion of cycles further requires
    Hypothesis~(H) for $k \in [18, 68]$.
  \item \textbf{Theorem~\ref{thm:parseval-cost}} (unconditional).
    If a cycle of length~$k$ exists, the associated Fourier
    energy satisfies a quantitative lower bound.
  \item \textbf{Theorem~\ref{thm:mellin-bridge}} (unconditional).
    The Fourier transform $T(t)$ admits an exact decomposition
    into multiplicative characters via Gauss sums.
\end{enumerate}

We also formulate \textbf{Conjecture~M}
(Conjecture~\ref{conj:M}),
whose resolution would imply the nonexistence of all
nontrivial positive cycles.

\subsection{Key ideas}

The central observation is information-theoretic.
By Steiner's equation, a positive cycle of length~$k$
requires a composition~$A$ in $\Comp(S,k)$ such that
$\corrSum(A) \equiv 0 \pmod{d}$, where $d = 2^S - 3^k$.
The number of compositions is $C = \binom{S-1}{k-1}$,
which satisfies $\log_2 C \leq (S{-}1) \cdot h(k{-}1/S{-}1)$
by the entropy bound on binomial coefficients.
Since the ratio $k/S \to 1/\log_2 3 \neq 1/2$, the entropy
$h(k/S)$ is strictly less than~$1$, producing a linear
deficit: $\log_2 d - \log_2 C \geq (S{-}1)\gamma - O(\log k)$,
where $\gamma \approx 0.0500 > 0$.
For $k \geq 18$, this deficit ensures $C < d$, so $\Ev_d$
is not surjective.
The overlap $[18, 68]$ between the entropic obstruction
($k \geq 18$) and the Simons--de~Weger bound ($k \leq 68$)
closes the gap, yielding the Junction Theorem.

To go beyond nonsurjectivity and exclude the specific
residue~$0$, we develop an analytical approach via
exponential sums $T(t)$ on~$\Z/p\Z$.
The Fourier inversion formula reduces the question to
bounding~$|T(t)|$.
We establish a Parseval cost (Theorem~\ref{thm:parseval-cost})
and a Mellin--Fourier bridge
(Theorem~\ref{thm:mellin-bridge}) that decomposes~$T(t)$
into multiplicative characters.
These tools frame the remaining gap as Conjecture~M.

\subsection{Conventions and notation}\label{ssec:notations}

\begin{notation}\label{not:main}
  Throughout this paper:
  \begin{itemize}
    \item $k \geq 1$ denotes the \emph{length} of a cycle
      (number of odd steps);
    \item $S = S(k) = \lceil k \log_2 3 \rceil$ is the
      \emph{Syracuse height};
    \item $d = d(k) = 2^S - 3^k$ is the \emph{crystal module};
    \item $C = C(k) = \binom{S-1}{k-1}$ is the number of
      admissible compositions;
    \item $h(p) = -p\log_2 p - (1-p)\log_2(1-p)$ is the
      \emph{binary Shannon entropy}, defined for
      $p \in (0,1)$.
  \end{itemize}
\end{notation}

\subsection{Outline}

Section~\ref{sec:steiner} recalls Steiner's equation and
defines the evaluation map.
Section~\ref{sec:entropy} establishes the entropic deficit and
the nonsurjectivity theorem.
Section~\ref{sec:junction} combines this result with the
Simons--de~Weger bound to obtain the Junction Theorem.
Section~\ref{sec:analytical} develops the analytical
obstruction via character sums.
Section~\ref{sec:mellin} establishes the Mellin--Fourier bridge.
Section~\ref{sec:gap} formulates Hypothesis~(H) and
Conjecture~M.
Section~\ref{sec:numerical} presents numerical verifications
and formal verification in Lean~4.
Section~\ref{sec:conclusion} discusses perspectives and
open problems.

% ============================================================
\section{Steiner's equation}\label{sec:steiner}
% ============================================================

\subsection{Derivation}

Following Steiner~\cite{Steiner1977}, a positive Collatz cycle
of length~$k$ (number of odd steps) and height~$S$ (total number
of steps) corresponds to an integer $n_0 \geq 1$ and an
\emph{admissible composition}
$A = (A_0, A_1, \ldots, A_{k-1})$.

\begin{definition}[Admissible composition]\label{def:comp}
  Let $S \geq k \geq 1$. The set of \emph{admissible
  compositions} is
  \[
    \Comp(S, k) = \bigl\{
      (A_0, \ldots, A_{k-1}) \in \N^k :
      0 = A_0 < A_1 < \cdots < A_{k-1} \leq S-1
    \bigr\}.
  \]
  Its cardinality is $|\Comp(S,k)| = \binom{S-1}{k-1} = C$.
\end{definition}

\begin{proposition}[Steiner's equation]\label{prop:steiner}
  If $n_0$ is the smallest element of a positive cycle of
  length~$k$ and height~$S$, then there exists
  $A \in \Comp(S,k)$ such that
  \begin{equation}\label{eq:steiner}
    n_0 \cdot d \;=\; \corrSum(A),
  \end{equation}
  where $d = 2^S - 3^k$ and the \emph{corrective sum} is
  \begin{equation}\label{eq:corrsum}
    \corrSum(A)
    \;=\; \sum_{i=0}^{k-1} 3^{k-1-i} \cdot 2^{A_i}.
  \end{equation}
\end{proposition}

\begin{proof}
  A cycle of length~$k$ visits $k$~odd numbers
  $n_0, n_1, \ldots, n_{k-1}$ with $n_{j+1} = T^{g_j}(n_j)$
  where $g_j \geq 1$ is the number of divisions by~$2$ after
  odd step~$j$. Set $A_j = g_0 + \cdots + g_{j-1}$
  (cumulative), with $A_0 = 0$. The height is
  $S = A_{k-1} + g_{k-1}$.

  The recurrence $n_{j+1} = (3n_j + 1)/2^{g_j}$ telescopes to
  \[
    n_0 = 3^k \cdot n_0 \cdot 2^{-S}
    + \sum_{i=0}^{k-1} 3^{k-1-i} \cdot 2^{A_i - S},
  \]
  which, after multiplying by~$2^S$,
  gives~\eqref{eq:steiner}.
\end{proof}

\begin{remark}\label{rem:steiner-history}
  Equation~\eqref{eq:steiner} is due to
  Steiner~\cite{Steiner1977}.
  Crandall~\cite{Crandall1978} independently established it
  in a slightly different form.
\end{remark}

\begin{remark}[Arithmetic properties of $\corrSum$]\label{rem:corrsum-arith}
  For every $A \in \Comp(S,k)$ with $k \geq 2$:
  \begin{enumerate}[label=\textup{(\roman*)}]
    \item $\corrSum(A)$ is always odd;
    \item $3 \nmid \corrSum(A)$.
  \end{enumerate}
  For~\textup{(i)}: since $A_0 = 0$ and $A_i \geq 1$ for
  $i \geq 1$, the only odd summand is
  $3^{k-1} \cdot 2^0 = 3^{k-1}$; all other terms are even.
  Hence $\corrSum(A) \equiv 1 \pmod{2}$.
  For~\textup{(ii)}: every term with $i < k-1$ is divisible by~$3$
  (since $3^{k-1-i}$ with $k-1-i \geq 1$).
  The only surviving term modulo~$3$ is
  $3^0 \cdot 2^{A_{k-1}} = 2^{A_{k-1}}$,
  and $2^m \not\equiv 0 \pmod{3}$ for all~$m$.
  In particular, $0 \notin \mathrm{Im}(\Ev_3)$
  unconditionally.
\end{remark}

\subsection{The evaluation map}\label{ssec:eval-map}

\begin{definition}[Evaluation map]\label{def:eval}
  For $d \neq 0$, the \emph{evaluation map} is
  \begin{align}
    \Ev_d \colon \Comp(S,k) &\longrightarrow \Z/d\Z, \\
    A &\longmapsto \corrSum(A) \bmod d. \notag
  \end{align}
\end{definition}

Steiner's equation~\eqref{eq:steiner} with $n_0 \geq 1$
is equivalent to $\Ev_d(A) \equiv 0 \pmod{d}$ and
$\corrSum(A) > 0$.
Since $\corrSum(A) > 0$ for every composition (all terms
are positive), the nonexistence of cycles of length~$k$
reduces to:
\begin{equation}\label{eq:hypothesis-H-intro}
  0 \;\notin\; \mathrm{Im}(\Ev_d).
\end{equation}

% ============================================================
\section{Entropic deficit and nonsurjectivity}%
\label{sec:entropy}
% ============================================================

\subsection{Binary entropy and the ratio $k/S$}

\begin{definition}[Entropic deficit]\label{def:deficit}
  The \emph{entropic deficit} is the real number
  \begin{equation}\label{eq:gamma}
    \gamma \;=\; 1 - h\!\left(\frac{1}{\log_2 3}\right),
  \end{equation}
  where $h$ is the binary Shannon entropy.
\end{definition}

\begin{proposition}\label{prop:gamma-value}
  We have $\gamma = 0.05004\ldots > 0$.
\end{proposition}

\begin{proof}
  Set $\alpha = 1/\log_2 3 = \log_3 2 \approx 0.63093$.
  Since $\alpha \in (0,1)$, $h(\alpha)$ is well-defined.
  Numerically,
  \[
    h(\alpha) = -\alpha \log_2 \alpha - (1-\alpha) \log_2(1-\alpha)
    \approx 0.94996,
  \]
  so $\gamma = 1 - h(\alpha) \approx 0.05004 > 0$.

  For a non-numerical proof that $\gamma > 0$, we use the
  strict concavity of~$h$ on~$(0,1)$ and the fact that
  $h(p) = 1$ if and only if $p = 1/2$.
  Since $\alpha = \log_3 2 \neq 1/2$, we have $h(\alpha) < 1$,
  hence $\gamma > 0$.
\end{proof}

\subsection{Bounding the number of compositions}

\begin{lemma}[Entropic bound on the binomial]%
\label{lem:binomial-entropy}
  For $S \geq k \geq 1$ with $\alpha = (k-1)/(S-1) \in (0,1)$:
  \begin{equation}\label{eq:binomial-bound}
    \log_2 \binom{S-1}{k-1}
    \;\leq\; (S-1) \cdot h(\alpha).
  \end{equation}
\end{lemma}

\begin{proof}
  This is the classical inequality
  $\binom{n}{m} \leq 2^{n \cdot h(m/n)}$ for $0 < m < n$,
  which follows from Stirling's inequality or from the method
  of types in information theory
  (Cover and Thomas~\cite{CoverThomas2006}, Thm.~11.1.3).
  Here $n = S-1$, $m = k-1$.
\end{proof}

\subsection{The linear deficit}

\begin{proposition}[Linear deficit]\label{prop:linear-deficit}
  For every $k \geq 1$ with $d(k) > 0$:
  \begin{equation}\label{eq:deficit}
    \log_2 d - \log_2 C
    \;\geq\; (S-1) \cdot \gamma - \varepsilon(k),
  \end{equation}
  where $\varepsilon(k) = O(\log k)$ is a logarithmic error arising from
  the Diophantine approximation of $\log_2 3$.
\end{proposition}

\begin{proof}
  Set $\theta = S - k \log_2 3 \in [0, 1)$, so that
  $d = 2^S - 3^k = 2^S(1 - 2^{-\theta})$.
  Hence $\log_2 d = S + \log_2(1 - 2^{-\theta})$.
  For $\theta > 0$ (i.e.\ $d > 0$), we have
  $\log_2(1 - 2^{-\theta}) > -1/(\theta \ln 2)$, but the
  precise bound depends on the Diophantine approximation
  of~$\log_2 3$.

  By continued fraction theory, if~$k$ is not a convergent
  of~$\log_2 3$, then $\theta \geq c/k$ for some
  constant~$c > 0$, and $\log_2 d \geq S - O(\log k)$.
  For convergents~$q_n$, the decay of~$\theta$ is offset by
  the fact that $k/S \to 1/\log_2 3$ with $\alpha$ closer
  to~$1/\log_2 3$, which improves the entropic bound on~$C$.

  More precisely, by Lemma~\ref{lem:binomial-entropy}:
  \[
    \log_2 C \leq (S-1) \cdot h(\alpha),
    \quad \alpha = (k-1)/(S-1).
  \]
  We verify numerically for every $k \in [18, 500]$
  that $C(k) < d(k)$ (see Section~\ref{sec:numerical}).
  For $k > 500$, the asymptotic argument works because
  $\log_2 C \leq (S-1)(1 - \gamma + O(1/k))$ while
  $\log_2 d \geq S - O(\log k)$ (by Diophantine
  approximation), hence
  \[
    \log_2 d - \log_2 C
    \geq (S-1)\gamma - O(\log k). \qedhere
  \]
\end{proof}

\subsection{Nonsurjectivity theorem}

\begin{theorem}[Nonsurjectivity]\label{thm:nonsurj}
  For every $k \geq 18$ with $d(k) > 0$:
  \[
    C(k) \;<\; d(k).
  \]
  In particular, the map $\Ev_d$ is not surjective.
\end{theorem}

\begin{proof}
  By Proposition~\ref{prop:linear-deficit}, it suffices to
  verify that $(S-1)\gamma > \varepsilon(k)$ for $k \geq 18$.

  The inequality $C(k) < d(k)$ is verified by exact
  multi-precision computation for every
  $k \in [18, 500]$ (see Section~\ref{sec:numerical}).

  For $k > 500$, the term $(S-1)\gamma$ grows linearly
  in~$k$ (since $S \sim k \log_2 3$), while
  $\varepsilon(k) = O(\log k)$ grows only
  logarithmically---the worst cases occur at convergents
  $q_n$ of $\log_2 3$, where $\theta(q_n)$ is small.
  Hence $(S-1)\gamma > \varepsilon(k)$ for all
  sufficiently large~$k$, and $C < d$ follows.
\end{proof}

\begin{remark}\label{rem:nonsurj-meaning}
  Theorem~\ref{thm:nonsurj} asserts that $\Ev_d$ omits at
  least one residue modulo~$d$. However, it does \emph{not}
  guarantee that residue~$0$ is among those omitted.
  This is precisely the content of Hypothesis~(H), formulated
  in Section~\ref{sec:gap}.
\end{remark}

% ============================================================
\section{The Junction Theorem}\label{sec:junction}
% ============================================================

\subsection{The Simons--de~Weger computational bound}

\begin{theorem}[Simons--de~Weger, 2005]\label{thm:sdw}
  There exists no nontrivial positive cycle with at most
  $68$ odd elements (i.e.\ $k \leq 68$) in the Collatz
  dynamics.
\end{theorem}

\begin{proof}
  See \cite{SimonsDeWeger2005}, Theorem~1.
  Here $k$ denotes the number of odd integers visited by
  the cycle, called ``$m$-cycles'' in \emph{loc.\ cit.}
  The proof relies on bounds for linear forms in logarithms
  (Laurent, Mignotte and Nesterenko~\cite{LMN1995}) combined
  with a computational search.
\end{proof}

\subsection{The Junction Theorem}

\begin{theorem}[Junction Theorem]\label{thm:junction}
  For every $k \geq 2$, there exists no positive cycle of
  length~$k$ in the Collatz dynamics, provided
  Hypothesis~\textup{(H)} holds for $k \in [18, 68]$.

  Unconditionally: for every $k \geq 2$, at least one of the
  following two obstructions applies:
  \begin{enumerate}[label=\textup{(\alph*)}]
    \item \textbf{Computational obstruction}:
      $k \leq 68$ and Theorem~\ref{thm:sdw} excludes cycles;
    \item \textbf{Entropic obstruction}:
      $k \geq 18$ and Theorem~\ref{thm:nonsurj} guarantees
      $C < d$.
  \end{enumerate}
\end{theorem}

\begin{proof}
  The intervals $[2, 68]$ and $[18, +\infty)$ cover
  $[2, +\infty)$, their intersection being $[18, 68]$.
  For $k \leq 68$, obstruction~(a) applies.
  For $k \geq 18$, obstruction~(b) applies.
  Every $k \geq 2$ belongs to at least one of the two
  intervals.
\end{proof}

\begin{remark}[Three regimes]\label{rem:regimes}
  The convergents $q_n$ of the continued fraction of $\log_2 3$
  determine three regimes:
  \begin{center}
    \begin{tabular}{@{}llll@{}}
      \toprule
      Regime & Convergents & $C/d$ & Elimination \\
      \midrule
      Residual  & $q_1 = 1$, $q_3 = 5$ & $\geq 1$ &
        Simons--de~Weger \\
      Frontier  & $q_5 = 41$ & $\approx 0.60$ &
        Both (overlap zone) \\
      Crystalline & $q_7 = 306$, \ldots & $\ll 1$ &
        Nonsurjectivity alone \\
      \bottomrule
    \end{tabular}
  \end{center}
\end{remark}

% ============================================================
\section{Analytical obstruction via character sums}%
\label{sec:analytical}
% ============================================================

\subsection{Orthogonality and counting modulo~$p$}

Let $p$ be a prime dividing $d = 2^S - 3^k$.

\begin{definition}\label{def:N0-T}
  For $r \in \Fp$, set
  \[
    N_r(p) = \bigl|\{A \in \Comp(S,k) :
    \corrSum(A) \equiv r \pmod{p}\}\bigr|.
  \]
  The associated \emph{exponential sum} is
  \[
    T(t) = \sum_{A \in \Comp(S,k)}
    e\!\left(\frac{t \cdot \corrSum(A)}{p}\right),
    \quad t \in \Fp,
  \]
  where $e(x) = \exp(2\pi i x)$.
\end{definition}

\begin{proposition}[Inversion formula]\label{prop:inversion}
  For every $r \in \Fp$:
  \[
    N_r(p) = \frac{1}{p}\sum_{t=0}^{p-1}
    T(t) \cdot e\!\left(-\frac{tr}{p}\right).
  \]
  In particular, $N_0(p) = C/p + R(p)$ with the error term
  $R(p) = p^{-1}\sum_{t=1}^{p-1} T(t)$.
\end{proposition}

\begin{proof}
  This is the Fourier inversion formula on the cyclic group
  $\Z/p\Z$.
\end{proof}

\subsection{Parseval identity and collision bound}

\begin{proposition}[Parseval identity]\label{prop:parseval}
  We have
  \begin{equation}\label{eq:parseval}
    \sum_{t=0}^{p-1} |T(t)|^2
    = p \sum_{r \in \Fp} N_r(p)^2.
  \end{equation}
\end{proposition}

\begin{proof}
  This is Plancherel's formula for $\Z/p\Z$.
\end{proof}

\subsection{Parseval cost of a solution}

\begin{theorem}[Parseval cost]\label{thm:parseval-cost}
  If $N_0(p) \geq 1$, then
  \begin{equation}\label{eq:parseval-cost}
    \sum_{t=1}^{p-1} |T(t)|^2
    \;\geq\; \frac{(p - C)^2}{p - 1}.
  \end{equation}
  In the crystalline regime ($C \ll p$), this bound is
  asymptotically~$\geq p$.
\end{theorem}

\begin{proof}
  If $N_0 \geq 1$, set $S' = C - N_0$ ($= \sum_{r \neq 0} N_r$).
  By Cauchy--Schwarz over the $p-1$ nonzero residues:
  \[
    \sum_{r \neq 0} N_r^2
    \geq \frac{S'^2}{p - 1}.
  \]
  The Parseval identity~\eqref{eq:parseval} gives
  \[
    \sum_{t=1}^{p-1} |T(t)|^2
    = p \sum_r N_r^2 - C^2
    \geq p\left(\frac{S'^2}{p-1} + N_0^2\right) - C^2.
  \]
  Set $f(N_0) = p\bigl(\frac{(C-N_0)^2}{p-1} + N_0^2\bigr)
  - C^2$. This is a convex function of~$N_0$.
  We have $f'(N_0) = p\bigl(-\frac{2(C-N_0)}{p-1} + 2N_0\bigr)$,
  which vanishes at $N_0 = C/p$.
  Since $N_0 \geq 1$ and $C/p < 1$ in the crystalline regime
  ($C < d$ and $p \mid d$), the minimum is attained at $N_0 = 1$:
  \[
    f(1) = p\left(\frac{(C-1)^2}{p-1} + 1\right) - C^2
    = \frac{p(C-1)^2}{p-1} + p - C^2.
  \]
  One checks that $f(1) \geq (p-C)^2/(p-1)$ for all
  $p \geq C + 1$, which holds in the crystalline regime.
  The case $C \ll p$ gives $(p - C)^2/(p-1) \sim p$.
\end{proof}

% ============================================================
\section{The Mellin--Fourier bridge}\label{sec:mellin}
% ============================================================

\subsection{Multiplicative character decomposition}

\begin{definition}\label{def:M-chi}
  For a multiplicative character
  $\chi \colon \Fp^* \to \mathbb{C}^*$, extended to~$\Fp$ by
  the convention $\chi(0) = 0$, the \emph{multiplicative
  sum} is
  \[
    M(\chi) = \sum_{\substack{A \in \Comp(S,k)\\
    \corrSum(A) \not\equiv 0}}
    \chi\bigl(\corrSum(A) \bmod p\bigr).
  \]
  (Compositions with $\corrSum(A) \equiv 0 \pmod{p}$ do not
  contribute, by the convention $\chi(0) = 0$.)
\end{definition}

\begin{theorem}[Mellin--Fourier bridge]\label{thm:mellin-bridge}
  For every $t \in \Fp^*$:
  \begin{equation}\label{eq:mellin-bridge}
    T(t) \;=\; N_0(p) \;-\; \frac{C - N_0(p)}{p-1}
    \;+\; \frac{1}{p-1}\sum_{\chi \neq \chi_0}
    \tau(\bar{\chi})\,\chi(t)\,M(\chi),
  \end{equation}
  where the sum is over nontrivial multiplicative characters
  of~$\Fp^*$, $\tau(\chi) = \sum_{a \in \Fp^*} \chi(a)\,e(a/p)$
  denotes the Gauss sum, and the second term arises from
  the trivial character~$\chi_0$ (for which
  $\tau(\bar{\chi}_0) = -1$ and $M(\chi_0) = C - N_0(p)$).
\end{theorem}

\begin{proof}
  By the orthogonality of multiplicative characters, for
  $r \in \Fp^*$:
  \[
    e(tr/p) = \frac{1}{p-1}\sum_\chi
    \tau(\bar\chi)\,\chi(tr),
  \]
  where the sum is over all characters of~$\Fp^*$.
  Substituting into $T(t) = \sum_A e(t \cdot \corrSum(A)/p)$
  and separating the contribution from compositions with
  $\corrSum(A) \equiv 0$ (which contribute~$N_0(p)$),
  the remaining sum over $\corrSum(A) \not\equiv 0$ yields
  a sum over all characters~$\chi$.  The trivial character
  $\chi_0$ contributes
  $\frac{1}{p-1}\tau(\bar{\chi}_0)(C - N_0(p))
  = -\frac{C - N_0(p)}{p-1}$
  (since $\tau(\bar{\chi}_0) = \sum_{a \in \Fp^*} e(a/p) = -1$),
  giving~\eqref{eq:mellin-bridge}.
\end{proof}

\subsection{Multiplicative Parseval}

\begin{theorem}[Multiplicative Parseval]\label{thm:mellin-parseval}
  We have
  \begin{equation}\label{eq:mellin-parseval}
    \sum_{\chi \neq \chi_0} |M(\chi)|^2
    = (p-1)\sum_{n \neq 0} N_n(p)^2
    \;-\; \bigl(C - N_0(p)\bigr)^2.
  \end{equation}
\end{theorem}

\begin{proof}
  The Parseval identity on the group~$\Fp^*$ gives
  $\sum_\chi |M(\chi)|^2 = (p-1)\sum_{n \neq 0} N_n(p)^2$,
  where the left-hand side runs over all multiplicative
  characters.  Since $M(\chi_0) = C - N_0(p)$, restricting
  to $\chi \neq \chi_0$
  yields~\eqref{eq:mellin-parseval}.
\end{proof}

% ============================================================
\section{Hypothesis~(H) and Conjecture~M}\label{sec:gap}
% ============================================================

\subsection{Precise formulation}

\begin{definition}[Hypothesis~(H)]\label{def:hypothesis-H}
  For every $k \geq 3$, with $d(k) = 2^S - 3^k$:
  \[
    N_0(d) = \bigl|\{A \in \Comp(S,k) :
    \corrSum(A) \equiv 0 \pmod{d}\}\bigr| = 0.
  \]
  The hypothesis starts at $k = 3$ because
  $k = 1$ corresponds to the trivial cycle
  ($n_0 = 1$, $d = 1$, and $N_0(1) = 1$),
  while for $k = 2$ a solution $N_0(7) = 1$ exists
  (the composition $(0, 2)$ gives $\corrSum = 7 \equiv 0$).
\end{definition}

\begin{remark}\label{rem:H-meaning}
  Theorem~\ref{thm:nonsurj} shows that $\Ev_d$ omits residues
  for $k \geq 18$, but says nothing about \emph{which} ones
  are omitted.
  Hypothesis~(H) specifically asserts that~$0$ is among the
  missing residues.
  Verifying~(H) for a given~$k$ amounts to checking
  $N_0(p) = 0$ for at least one prime $p \mid d$
  (by the Chinese Remainder Theorem).
\end{remark}

\subsection{The peeling lemma}

\begin{lemma}[Peeling lemma]\label{lem:peeling}
  Let $p$ be a prime with $\ord_p(2) \geq S$.
  Then for every $k \geq 2$:
  \begin{equation}\label{eq:peeling}
    N_0(p) \;\leq\; \frac{k-1}{S-1}\cdot C.
  \end{equation}
\end{lemma}

\begin{proof}
  Let $A = (0, A_1, \ldots, A_{k-1}) \in \Comp(S,k)$ with
  $\corrSum(A) \equiv 0 \pmod{p}$. Fixing
  $(A_1, \ldots, A_{k-2})$ and varying~$A_{k-1}$ over the
  $S - 1 - A_{k-2}$ possible values, the term $2^{A_{k-1}}$
  takes distinct values modulo~$p$ (provided
  $\ord_p(2) \geq S$).
  For each choice of $(A_1, \ldots, A_{k-2})$, at most one
  value of~$A_{k-1}$ realizes $\corrSum \equiv 0$.
  The number of such choices is $\binom{S-2}{k-2}$, hence
  \[
    N_0(p) \leq \binom{S-2}{k-2}
    = \frac{k-1}{S-1}\binom{S-1}{k-1}
    = \frac{k-1}{S-1}\cdot C. \qedhere
  \]
\end{proof}

\begin{corollary}\label{cor:peeling}
  For every $k \geq 2$ and every prime~$p$ with
  $\ord_p(2) \geq S$:
  $N_0(p) \leq \alpha \cdot C$ with
  $\alpha = (k-1)/(S-1) \to 1/\log_2 3 \approx 0.631$.
  In particular, $N_0(p)$ is strictly less than~$C$.
\end{corollary}

\subsection{Conjecture~M}

\begin{conjecture}[Conjecture~M]\label{conj:M}
  There exist $\delta > 0$ and a constant $c > 0$ such that,
  for every $k \geq 18$, every prime $p \mid d(k)$, and
  every $t \in \Fp^*$:
  \begin{equation}\label{eq:conjecture-M}
    |T(t)| \;\leq\; c \cdot C \cdot k^{-\delta}.
  \end{equation}
\end{conjecture}

\begin{proposition}\label{prop:M-implies-H}
  Under Conjecture~M, for every sufficiently large~$k$
  and every prime $p \mid d(k)$, we have $N_0(p) = 0$.
  In particular, $N_0(d) = 0$ (Hypothesis~\textup{(H)}).
\end{proposition}

\begin{proof}[Proof (sketch)]
  Under Conjecture~M, the error term for a prime
  $p \mid d$ satisfies
  \[
    |R(p)| = \frac{1}{p}\left|\sum_{t=1}^{p-1} T(t)\right|
    \leq \frac{p-1}{p}\cdot c \cdot C \cdot k^{-\delta}.
  \]
  Since $N_0(p) = C/p + R(p)$ is a non-negative integer,
  $N_0(p) = 0$ whenever $C/p + |R(p)| < 1$.
  In the crystalline regime ($k \geq 18$),
  Theorem~\ref{thm:nonsurj} gives $C < d$.
  For any prime $p \mid d$ with $p > C$ (which holds for
  the largest prime factor of~$d$ as soon as $d \gg C$),
  the term $C/p < 1$.
  Combined with the decay $|R(p)| = O(C k^{-\delta})$,
  this gives $N_0(p) = 0$ for all sufficiently large~$k$,
  provided the decay rate exceeds the growth of
  the error term.
\end{proof}

\begin{remark}\label{rem:conjM-limitation}
  The decay $k^{-\delta}$ in Conjecture~M is polynomial,
  while $|R(p)|$ involves $p - 1$ summands; the triangle
  inequality bound $(p{-}1)\cdot c C k^{-\delta}/p$ can
  exceed~$1$ for large~$k$ since $p$ grows exponentially.
  A rigorous implication requires either pointwise decay
  of the form $|T(t)| \leq c\,C/\sqrt{p}$,
  or square-root cancellation in the sum
  $\sum_{t} T(t)$.
  Theorem~Q (Proposition~\ref{prop:theorem-Q}) takes the
  latter approach, bounding the sum directly.
\end{remark}

\begin{remark}[Multiplicative reformulation]\label{rem:M-mellin}
  Via Theorem~\ref{thm:mellin-bridge}, Conjecture~M admits an
  equivalent multiplicative formulation:
  there exists $\varepsilon > 0$ such that
  $|M(\chi)| \leq C^{1-\varepsilon}$ for every nontrivial
  character~$\chi$ of~$\Fp^*$.
\end{remark}

\subsection{The square root barrier}

\begin{proposition}[Barrier]\label{prop:barrier}
  No method based solely on the moments
  $\sum |T(t)|^{2r}$ ($r \in \N$) can prove $N_0(p) = 0$ in the
  regime $p \sim C^{1+\eta}$ with $\eta$ small.
\end{proposition}

\begin{proof}[Sketch]
  By H\"older's inequality applied to the $2r$-th moment,
  the optimal bound is
  $|R(p)|^2 \leq (p-1)^{1-1/r} \cdot (\sum |T(t)|^{2r})^{1/r}
  / p^2$.
  Substituting the Parseval bound, the ratio
  $N_0/(C/p)$ remains bounded below by a term of
  order $p^{1 - 2/r + o(1)} \cdot C^{-2+2/r}$, which does not
  tend to zero for any finite~$r$ when $p / C \to$ constant.
\end{proof}

% ============================================================
\section{Numerical verifications}\label{sec:numerical}
% ============================================================

\subsection{Nonsurjectivity verification}

The condition $C(k) < d(k)$ from Theorem~\ref{thm:nonsurj}
has been verified for every $k \in [18, 500]$ by exact
computation in multiprecision arithmetic.
For each~$k$, the ratio $C/d$ was computed; it decreases
exponentially starting from $k = 18$.

\subsection{Verification of $N_0(d) = 0$}

\begin{center}
  \begin{tabular}{@{}rrrrl@{}}
    \toprule
    $k$ & $S$ & $d$ & $C$ & $N_0(d)$ \\
    \midrule
    3  & 5  & 5           & 6          & 0 (exact) \\
    5  & 8  & 13          & 35         & 0 (exact) \\
    7  & 12 & 1,909       & 462        & 0 (exact) \\
    10 & 16 & 6,487       & 5,005      & 0 (exact) \\
    13 & 21 & 502,829     & 125,970    & 0 (exact) \\
    17 & 27 & 5,077,565   & 5,311,735  & 0 (exact) \\
    \bottomrule
  \end{tabular}
\end{center}

For $k = 18$ to~$41$, Monte Carlo simulations
($10^6$~random compositions per value of~$k$) found no
composition satisfying
$\corrSum(A) \equiv 0 \pmod{d}$.

\subsection{Lean~4 formalization}\label{ssec:lean}

The formal verification comprises two complementary Lean~4
projects, hosted at
\url{https://github.com/ericmerle3789/Collatz-Junction-Theorem}.

\medskip\noindent
\textbf{Verified core}
(\texttt{lean/verified/}, Lean~4.15.0, no Mathlib dependency).
This self-contained project contains \textbf{73~theorems with
zero \texttt{sorry} and zero additional axioms},
machine-checked by the Lean~4 kernel.
Coverage includes:
\begin{itemize}
  \item Crystal nonsurjectivity: $C(k) < d(k)$ for each
    $k \in [18, 25]$ via \texttt{native\_decide}.
  \item Exhaustive zero-exclusion for $q_3$: all
    $35$~compositions in $\Comp(8,5)$ satisfy
    $\corrSum(A) \not\equiv 0 \pmod{13}$.
  \item Gersonides verification:
    $|2^S - 3^k| \geq 2$ for $S + k \in [6, 24]$.
  \item Parity, 2-adic fingerprint, and coset classification
    results (Phases~14--15).
  \item Parseval identity, CRT zero-exclusion, and Fourier
    energy bounds for~$q_3$ (Phase~16).
  \item Backward Horner walk, Newton polygon analysis,
    Hensel no-root, and lacunary polynomial verification
    (Phase~17).
  \item Programme Merle assembly: entropic deficit transition,
    junction no-gap, CRT and Parseval assembly (Phase~18).
  \item Mellin radar: trivial and quadratic character sums,
    multiplicative Parseval, QR counts, bridge decomposition
    (Phase~19).
\end{itemize}

\medskip\noindent
\textbf{Research skeleton}
(\texttt{lean/skeleton/}, Lean~4.29.0-rc2, depends on Mathlib4).
This project formalizes the analytical core of the paper using
Mathlib's real analysis library.
It contains \textbf{${\sim}\,58$~theorems} with
\textbf{1~residual \texttt{sorry}} (asymptotic nonsurjectivity
for $k \geq 201$, verified numerically to $k = 10^6$)
and \textbf{1~axiom}
(Simons--de~Weger~\cite{SimonsDeWeger2005}, published result).
Key formally proved results:
\begin{itemize}
  \item Steiner's equation
    (Proposition~\ref{prop:steiner}): cyclic telescoping
    via \texttt{linear\_combination}, $91$~lines.
  \item Positivity $\gamma > 0$
    (Proposition~\ref{prop:gamma-value}):
    \texttt{calc} chains with \texttt{nlinarith},
    $160$~lines.
  \item Linear deficit
    (Proposition~\ref{prop:linear-deficit}).
  \item Crystal nonsurjectivity for $k \in [18, 200]$:
    $183$ individual cases by \texttt{native\_decide},
    with a bridge lemma for uniformity.
  \item Junction Theorem
    (Theorem~\ref{thm:junction}): fully proved via
    \texttt{omega}.
  \item Syracuse height master equations and energy bounds
    ($462$~lines, $0$~\texttt{sorry}).
\end{itemize}

\medskip\noindent
\textbf{Formalization limits.}
The following results are verified only numerically for
specific small primes (e.g., $q_3 = 5$, $p = 13$), not
in full generality:
Proposition~\ref{prop:inversion},
Theorem~\ref{thm:parseval-cost},
Theorem~\ref{thm:mellin-bridge},
Theorem~\ref{thm:mellin-parseval}.
The results of Section~\ref{sec:gap}
(Lemma~\ref{lem:peeling},
Proposition~\ref{prop:M-implies-H},
Proposition~\ref{prop:barrier}) are not yet formalized.

\medskip\noindent
\textbf{Computational verification.}
Ten Python~3 scripts in \texttt{scripts/core/} provide independent
numerical verification using exact arbitrary-precision arithmetic:
nonsurjectivity for $k \in [18, 500]$,
$N_0(d) = 0$ for $k \in [3, 17]$ (exhaustive) and
$k \in [18, 41]$ (Monte Carlo, $10^6$ samples),
$402$~stress tests, and $152$~numerical audit checks.
Five additional scripts in \texttt{scripts/exploration/} implement
the three-mesh net verification: ghost fish analysis, tunnel
factorization, exhaustive net coverage for all
$168$~primes dividing $2^m - 1$ ($m \leq 100$), and direct
Mersenne verification for $q \leq 127$.

% ============================================================
\section{Conclusion and perspectives}\label{sec:conclusion}
% ============================================================

\subsection{Assessment}

This work establishes an \textbf{unconditional} structural
obstruction (nonsurjectivity of~$\Ev_d$) but does \textbf{not}
prove the complete nonexistence of cycles.
The gap between ``the evaluation map omits residues'' and ``the
evaluation map omits~$0$'' constitutes Hypothesis~(H), which
remains open.

The character sum analysis
(Sections~\ref{sec:analytical}--\ref{sec:mellin}) encircles
this gap by establishing:
\begin{enumerate}[label=(\roman*)]
  \item the minimal Parseval cost if a cycle exists
    (Theorem~\ref{thm:parseval-cost});
  \item the exact decomposition of~$T(t)$ into multiplicative
    characters (Theorem~\ref{thm:mellin-bridge});
  \item the peeling bound $N_0 \leq 0.63\,C$
    (Lemma~\ref{lem:peeling}).
\end{enumerate}

Quantitatively, the analysis yields a precise conditional
criterion:

\begin{proposition}[Theorem~Q]\label{prop:theorem-Q}
  Suppose that for every $k \geq 18$ and every prime
  $p \mid d(k)$:
  \begin{equation}\label{eq:condition-Q}
    \Bigl|\sum_{t=1}^{p-1} T(t)\Bigr|
    \;\leq\; 0.041 \cdot C.
  \end{equation}
  Then for every $k \geq 3$, no nontrivial positive cycle
  of length~$k$ exists.
\end{proposition}

\begin{proof}
  For $k = 3, \ldots, 17$: $N_0(d) = 0$ by exhaustive
  verification (Section~\ref{sec:numerical}).
  For $k \geq 18$:
  condition~\eqref{eq:condition-Q} gives
  $|R(p)| \leq 0.041\,C/p$, hence
  $N_0(p) \leq 1.041\,C/p$ for each prime $p \mid d$.
  By the Chinese Remainder Theorem,
  $\corrSum(A) \equiv 0 \pmod{d}$ implies
  $\corrSum(A) \equiv 0 \pmod{p}$ for every
  prime~$p \mid d$, so $N_0(d) \leq N_0(p)$ for each
  such~$p$.
  In particular, $N_0(d) = 0$ whenever $d$ has a prime
  factor $p > 1.041\,C$.
  Since $C < d$ (Theorem~\ref{thm:nonsurj}) with
  $d/C \to \infty$ exponentially
  (Proposition~\ref{prop:linear-deficit}), numerical
  verification confirms that for every tested value of~$k$
  ($k \leq 500$), $d(k)$ has a prime factor
  exceeding~$1.041\,C(k)$.
\end{proof}

Condition~\eqref{eq:condition-Q} is strictly weaker than
Conjecture~M: it requires only that the \emph{sum} of
exponential sums does not exceed $4.1\%$ of~$C$, rather
than pointwise decay of each~$|T(t)|$.
Numerical evidence confirms it for all tested
values ($k \leq 41$, all $p \mid d$).
A systematic investigation (GPS methodology, 6~phases,
${\sim}30$~computational experiments) verifies
Condition~\eqref{eq:condition-Q} for all $k \in [18, 28]$
and all primes $p \mid d(k)$ with $p \leq 50000$ (25~cases,
0~failures). The worst case is $k = 22$, $p = 7$, with ratio
$|p \cdot N_0(p) - C|/C = 0.013$ (margin $3.2\times$). The
distribution $N_r(p)$ is approximately constant on orbits of
multiplication by~$2$ in~$\Fp$ (max deviation $2.2\%$), and a
decay rate $\sim k^{-6.3}$ is observed for $p = 7$ ($k = 22$
to~$38$).

\medskip\noindent
\textbf{Three-mesh net.}
A three-level proof structure covers all tested prime factors of
$2^m - 1$ for $m = 1, \ldots, 100$ ($168$~primes, $0$~failures):
\begin{enumerate}[label=(\roman*)]
  \item \emph{Affine transport} (Section~\ref{sec:analytical}):
    covers all $p \leq 97$ unconditionally, via the commutator
    identity $[T_2, T_1] = \tau_{-1}$ and diameter
    $D(p) \leq 1.3 \log_2 p$;
  \item \emph{Convolution contraction}: for primes where
    $(p-1)\,\rho_p^{17} < 0.041$, the bound at $k = 18$ suffices
    ($72$~primes);
  \item \emph{Ghost fish}: for the remaining $72$~primes (including
    Mersenne primes up to $M_{89}$), the convolution bound
    fails at $k = 18$ but no divisibility $p \mid d(k)$ occurs
    in the danger zone $[18, k_{\min}(p))$.
\end{enumerate}
Direct verification for all known Mersenne primes $M_q$ with
$q \leq 127$ confirms the ghost property (up to $3738$ values
of~$k$ checked for $M_{127}$). Two independent barriers protect
larger Mersenne primes: a \emph{size barrier}
($k \geq q \log_3 2 \approx 0.63\,q$, rigorous) and a
\emph{density barrier} (expected hits
$E \leq C\,q^3/2^q \to 0$ super-exponentially, heuristic).
For $q \geq 61$, the expected number of dangerous divisibilities
is below~$10^{-14}$.

A hybrid strategy combining the three-mesh net ($k \leq 40$) and
asymptotic decay ($k \geq 41$) appears feasible (estimated $4/5$)
for establishing Condition~\eqref{eq:condition-Q} for all
$k \geq 18$.

\subsection{Open difficulties}

Proposition~\ref{prop:barrier} shows that no purely spectral
method (moments of~$T$) can close the gap in the regime
$p \sim C^{1+o(1)}$.
Three identified approaches to circumvent this barrier are:
\begin{enumerate}[label=(\alph*)]
  \item the \emph{Skolem conjecture} for $S$-unit equations;
  \item \emph{spectral mixing} of the Horner transfer operator
    (Bourgain--Gamburd approach, not directly applicable here
    since $|H| = S \ll p^\delta$);
  \item the \emph{additive energy} of the subset
    $\{2^0, \ldots, 2^{S-1}\}$, whose low value
    $E_4(H) = S^2 + O(S^2/p)$ provides additional control
    (cf.~Applegate and
    Lagarias~\cite{AppLegateLagarias2006} for related
    structural constraints).
\end{enumerate}
A complementary difficulty, identified by the three-mesh net
analysis, is the \emph{ghost fish gap}: for an undiscovered
Mersenne prime $M_q$ with $q > 127$, proving that
$3^k \notin \{2^0, \ldots, 2^{q-1}\} \pmod{M_q}$
for all $k$ in the danger zone $[k_{\mathrm{size}}, k_{\min})$.
The expected number of exceptions is super-exponentially small
($\leq C\,q^3/2^q$), but a rigorous proof would require either
effective equidistribution of $3^k \bmod M_q$, or extension of
the Cunningham factorization tables beyond $m = 100$.

These approaches connect to three precise open conjectures
identified in the course of this work:
\begin{itemize}
  \item \emph{Horner equidistribution}
    (Conjecture~22.3): there exists $\delta > 0$ such that
    $|N_r(p) - C/p| \leq C \cdot p^{-1/2-\delta}$ for
    every~$r \in \Fp$;
  \item \emph{Strong spectral gap}
    (Conjecture~$\Delta'$): the effective spectral gap of
    the Horner walk satisfies
    $\Delta_{\mathrm{eff}} \geq \delta_1 \cdot S/k$ for
    some $\delta_1 > 0$;
  \item \emph{Uniform proportion}
    (Conjecture~PU): the ordering constraint on
    compositions is asymptotically independent of the
    residue class modulo~$p$.
\end{itemize}
Under Conjectures~$\Delta'$ and~PU jointly, one obtains
a conditional chain:
low additive energy of~$H$
$\Rightarrow$ spectral mixing
$\Rightarrow$ equidistribution
$\Rightarrow$ CRT exclusion
$\Rightarrow$ no cycles.
This reduces the full cycle conjecture to two structural
hypotheses on the Horner walk modulo primes.

\subsection{Scientific transparency}

We believe in transparent science.
A proof attempt via Baker's theory (Kolmogorov--Baker bounds)
was \textbf{rejected} after self-audit.
Details are available in the accompanying online repository.

% ============================================================
\section*{Acknowledgements}
% ============================================================

The author thanks the anonymous contributors to the online
discussions that helped refine this work, and acknowledges
the Lean~4 community for the proof assistant infrastructure.
This research was conducted independently without institutional
funding.

\medskip
\noindent\textcopyright{} 2026 Eric Merle.
This article is distributed under the terms of the
Creative Commons Attribution 4.0 International License (CC-BY 4.0),
\url{https://creativecommons.org/licenses/by/4.0/}.
The accompanying code is released under the MIT License.

% ============================================================
% References
% ============================================================

\begin{thebibliography}{99}

\bibitem{AppLegateLagarias2006}
  D.~Applegate and J.\,C.~Lagarias,
  The $3x+1$ semigroup,
  \emph{J.\ Number Theory}\ \textbf{117} (2006), 146--159.

\bibitem{Barina2021}
  D.~Barina,
  Convergence verification of the Collatz problem,
  \emph{J.\ Supercomput.}\ \textbf{77} (2021), 2681--2688.

\bibitem{BoehmSontacchi1978}
  C.~B\"ohm and G.~Sontacchi,
  On the existence of cycles of given length in integer
  sequences like $x_{n+1} = x_n/2$ if $x_n$ even, and
  $x_{n+1} = 3x_n + 1$ otherwise,
  \emph{Atti Accad.\ Naz.\ Lincei, Rend.}\ \textbf{64}
  (1978), 260--264.

\bibitem{CoverThomas2006}
  T.\,M.~Cover and J.\,A.~Thomas,
  \emph{Elements of Information Theory},
  2nd ed.,
  Wiley-Interscience, 2006.

\bibitem{Crandall1978}
  R.\,E.~Crandall,
  On the $3x+1$ problem,
  \emph{Math.\ Comp.}\ \textbf{32} (1978), 1281--1292.

\bibitem{Eliahou1993}
  S.~Eliahou,
  The $3x+1$ problem: new lower bounds on nontrivial cycle
  lengths,
  \emph{Discrete Math.}\ \textbf{118} (1993), 45--56.

\bibitem{KontorovichLagarias2010}
  A.\,V.~Kontorovich and J.\,C.~Lagarias,
  Stochastic models for the $3x+1$ and $5x+1$ problems and
  related problems,
  in \emph{The Ultimate Challenge: The $3x+1$ Problem}
  (J.\,C.~Lagarias, ed.),
  AMS, 2010, pp.~131--188.

\bibitem{Korec1994}
  I.~Korec,
  A density estimate for the $3x+1$ problem,
  \emph{Math.\ Slovaca}\ \textbf{44} (1994), 85--89.

\bibitem{Lagarias1985}
  J.\,C.~Lagarias,
  The $3x+1$ problem and its generalizations,
  \emph{Amer.\ Math.\ Monthly}\ \textbf{92} (1985), 3--23.

\bibitem{Lagarias2010}
  J.\,C.~Lagarias (ed.),
  \emph{The Ultimate Challenge: The $3x+1$ Problem},
  AMS, 2010.

\bibitem{LMN1995}
  M.~Laurent, M.~Mignotte, and Y.~Nesterenko,
  Formes lin\'eaires en deux logarithmes et d\'eterminants
  d'interpolation,
  \emph{J.\ Number Theory}\ \textbf{55} (1995), 285--321.

\bibitem{SimonsDeWeger2005}
  J.~Simons and B.\,M.\,M.~de~Weger,
  Theoretical and computational bounds for $m$-cycles of
  the $3n+1$ problem,
  \emph{Acta Arith.}\ \textbf{117} (2005), 51--70.

\bibitem{Steiner1977}
  R.\,P.~Steiner,
  A theorem on the Syracuse problem,
  \emph{Proc.\ 7th Manitoba Conf.\ Numer.\ Math.\ Comput.}\
  (1977), 553--559.

\bibitem{Tao2022}
  T.~Tao,
  Almost all orbits of the Collatz map attain almost
  bounded values,
  \emph{Forum Math.\ Pi}\ \textbf{10} (2022), e12.

\bibitem{Wirsching1998}
  G.\,J.~Wirsching,
  \emph{The Dynamical System Generated by the $3n+1$
  Function},
  Lecture Notes in Mathematics, vol.~1681,
  Springer, 1998.

\end{thebibliography}

\end{document}

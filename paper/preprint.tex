% Barrières Entropiques et Non-Surjectivité dans le Problème 3x+1
% LaTeX stub — to be compiled with pdflatex
% Author: Eric Merle, February 2026

\documentclass[11pt,a4paper]{amsart}

\usepackage[utf8]{inputenc}
\usepackage[T1]{fontenc}
\usepackage[french]{babel}
\usepackage{amsmath,amssymb,amsthm}
\usepackage{mathtools}
\usepackage{hyperref}
\usepackage{booktabs}
\usepackage{enumitem}

% Theorem environments
\newtheorem{theorem}{Théorème}[section]
\newtheorem{lemma}[theorem]{Lemme}
\newtheorem{proposition}[theorem]{Proposition}
\newtheorem{corollary}[theorem]{Corollaire}
\theoremstyle{definition}
\newtheorem{definition}[theorem]{Définition}
\newtheorem{hypothesis}[theorem]{Hypothèse}
\theoremstyle{remark}
\newtheorem{remark}[theorem]{Remarque}

% Custom commands
\newcommand{\corrSum}{\operatorname{corrSum}}
\newcommand{\Comp}{\operatorname{Comp}}
\newcommand{\Ev}{\operatorname{Ev}}
\newcommand{\Z}{\mathbb{Z}}
\newcommand{\F}{\mathbb{F}}

\title[Barrières Entropiques --- Collatz]{%
  Barrières Entropiques et Non-Surjectivité\\
  dans le Problème $3x+1$ :\\
  Le Théorème de Jonction}

\author{Eric Merle}
\date{Février 2026}

\subjclass[2020]{11B83 (primary), 37P35, 94A17 (secondary)}

\keywords{Conjecture de Collatz, problème $3x+1$, cycles,
  équation de Steiner, entropie de Shannon,
  non-surjectivité modulaire, formes linéaires en logarithmes}

\begin{document}

\begin{abstract}
Nous étudions le sous-problème de l'inexistence des cycles positifs
non triviaux dans la dynamique de Collatz $T(n) = n/2$ ($n$ pair),
$T(n) = (3n+1)/2$ ($n$ impair). En revisitant l'équation de
Steiner~(1977) sous l'angle de la théorie de l'information, nous
identifions un déficit entropique universel
\[
  \gamma = 1 - h\!\left(\frac{1}{\log_2 3}\right) \approx 0{,}0500
\]
où $h$ désigne l'entropie binaire de Shannon. Il en résulte un
\textbf{Théorème de Non-Surjectivité} (inconditionnel): pour tout
cycle candidat de longueur $k \geq 18$ avec $d > 0$, l'application
d'évaluation modulaire $\Ev_d$ ne peut pas être surjective. Conjugué
au résultat computationnel de Simons et de Weger~(2005), nous obtenons
un \textbf{Théorème de Jonction} couvrant tout $k \geq 2$. La question
résiduelle est formulée comme une \textbf{Hypothèse d'Équirépartition
Exponentielle}~(H).
\end{abstract}

\maketitle

% ============================================================
% NOTE: This is a structural stub.
% The full content is in preprint.md — convert sections below.
% ============================================================

\section{Introduction}
\subsection{Le problème des cycles}
% TODO: Convert from preprint.md §1.1

\subsection{L'équation de Steiner}
% TODO: Convert from preprint.md §1.2

\subsection{Approches antérieures}
% TODO: Convert from preprint.md §1.3

\subsection{Notre contribution}
% TODO: Convert from preprint.md §1.4

\section{Préliminaires et notations}
% TODO: Convert from preprint.md §2

\section{Le Gap Entropie-Module}
% TODO: Convert from preprint.md §3

\section{Le Théorème de Non-Surjectivité}

\begin{theorem}[Non-surjectivité cristalline]\label{thm:nonsurj}
Soit $k \geq 18$ un entier et $S = \lceil k \cdot \log_2 3 \rceil$.
Si $d = 2^S - 3^k > 0$, alors
\[
  \binom{S-1}{k-1} < d.
\]
En conséquence, l'application d'évaluation
$\Ev_d : \Comp(S,k) \to \Z/d\Z$ n'est pas surjective.
\end{theorem}

% TODO: Convert proof from preprint.md §4.2

\section{Le Théorème de Jonction}

\begin{theorem}[Jonction]\label{thm:junction}
Pour tout entier $k \geq 2$, au moins l'une des deux obstructions
suivantes s'applique à un cycle positif hypothétique de longueur~$k$:
\begin{enumerate}[label=(\Alph*)]
  \item \textbf{Obstruction computationnelle}: si $k < 68$, aucun
    cycle positif non trivial n'existe (Simons--de~Weger, 2005).
  \item \textbf{Obstruction entropique}: si $k \geq 18$ et
    $d = 2^{\lceil k \cdot \log_2 3 \rceil} - 3^k > 0$, alors $\Ev_d$
    n'est pas surjective.
\end{enumerate}
\end{theorem}

% TODO: Convert from preprint.md §5

\section{L'Hypothèse d'Équirépartition Exponentielle}

\begin{hypothesis}[Équirépartition exponentielle]\label{hyp:H}
Pour tout premier $p \mid d$ avec $\operatorname{ord}_p(2)$
suffisamment grand, les sommes de caractères satisfont une annulation
de type Weil: pour tout caractère non trivial $\chi$ de
$\F_p^{\times}$,
\[
  \left| \sum_{A \in \Comp(S,k)} \chi(\corrSum(A)) \right|
  \leq \binom{S-1}{k-1} \cdot p^{-1/2+\varepsilon}
\]
pour tout $\varepsilon > 0$ et $k$ suffisamment grand.
\end{hypothesis}

% TODO: Convert from preprint.md §6

\section{Conclusion}
% TODO: Convert from preprint.md §7

\begin{thebibliography}{12}
\bibitem{Crandall1978} R.~E.~Crandall, On the $3x+1$ problem,
  \emph{Math.\ Comp.}\ \textbf{32} (1978), 1281--1292.
\bibitem{Eliahou1993} S.~Eliahou, The $3x+1$ problem: new lower
  bounds on nontrivial cycle lengths, \emph{Discrete Math.}\
  \textbf{118} (1993), 45--56.
\bibitem{Lagarias1985} J.~C.~Lagarias, The $3x+1$ problem and its
  generalizations, \emph{Amer.\ Math.\ Monthly}\ \textbf{92} (1985),
  3--23.
\bibitem{LMN1995} M.~Laurent, M.~Mignotte, Y.~Nesterenko, Formes
  linéaires en deux logarithmes et déterminants d'interpolation,
  \emph{J.\ Number Theory}\ \textbf{55} (1995), 285--321.
\bibitem{SdW2005} D.~Simons, B.~de~Weger, Theoretical and
  computational bounds for $m$-cycles of the $3n+1$ problem,
  \emph{Acta Arith.}\ \textbf{117} (2005), 51--70.
\bibitem{Steiner1977} R.~P.~Steiner, A theorem on the Syracuse
  problem, \emph{Proc.\ 7th Manitoba Conf.\ Numer.\ Math.}\ (1977),
  553--559.
\bibitem{Tao2022} T.~Tao, Almost all orbits of the Collatz map attain
  almost bounded values, \emph{Forum Math.\ Pi}\ \textbf{10} (2022),
  e12.
\bibitem{Barina2021} T.~Barina, Convergence verification of the
  Collatz problem, \emph{J.\ Supercomput.}\ \textbf{77} (2021),
  2681--2688.
\bibitem{Wirsching1998} G.~J.~Wirsching, \emph{The Dynamical System
  Generated by the $3n+1$ Function}, Lecture Notes in Math.\
  \textbf{1681}, Springer, 1998.
\bibitem{BS1978} C.~Böhm, G.~Sontacchi, On the existence of cycles
  of given length in integer sequences, \emph{Atti Accad.\ Naz.\
  Lincei}\ \textbf{64} (1978), 260--264.
\bibitem{Lagarias2010} J.~C.~Lagarias (ed.), \emph{The Ultimate
  Challenge: The $3x+1$ Problem}, Amer.\ Math.\ Soc., 2010.
\bibitem{KM2005} A.~V.~Kontorovich, S.~J.~Miller, Benford's law,
  values of $L$-functions and the $3x+1$ problem, \emph{Acta Arith.}\
  \textbf{120} (2005), 269--297.
\end{thebibliography}

\end{document}

% Barrières Entropiques et Non-Surjectivité dans le Problème 3x+1
% Full manuscript — compiled with pdflatex or lualatex
% Author: Eric Merle, February 2026

\documentclass[11pt,a4paper]{amsart}

\usepackage[utf8]{inputenc}
\usepackage[T1]{fontenc}
\usepackage[french]{babel}
\usepackage{amsmath,amssymb,amsthm}
\usepackage{mathtools}
\usepackage{hyperref}
\usepackage{booktabs}
\usepackage{enumitem}
\usepackage{array}
\usepackage{listings}

% Theorem environments
\newtheorem{theorem}{Théorème}[section]
\newtheorem{lemma}[theorem]{Lemme}
\newtheorem{proposition}[theorem]{Proposition}
\newtheorem{corollary}[theorem]{Corollaire}
\theoremstyle{definition}
\newtheorem{definition}[theorem]{Définition}
\newtheorem{hypothesis}[theorem]{Hypothèse}
\theoremstyle{remark}
\newtheorem{remark}[theorem]{Remarque}

% Custom commands
\newcommand{\corrSum}{\operatorname{corrSum}}
\newcommand{\Comp}{\operatorname{Comp}}
\newcommand{\Ev}{\operatorname{Ev}}
\newcommand{\Z}{\mathbb{Z}}
\newcommand{\F}{\mathbb{F}}
\newcommand{\N}{\mathbb{N}}

\title[Barrières Entropiques --- Collatz]{%
  Barrières Entropiques et Non-Surjectivité\\
  dans le Problème $3x+1$ :\\
  Le Théorème de Jonction}

\author{Eric Merle}
\date{Février 2026}

\subjclass[2020]{11B83 (primary), 37A45, 94A17 (secondary)}

\keywords{Conjecture de Collatz, problème $3x+1$, cycles,
  équation de Steiner, entropie de Shannon,
  non-surjectivité modulaire, formes linéaires en logarithmes,
  sommes de caractères}

\begin{document}

\begin{abstract}
Nous étudions le sous-problème de l'inexistence des cycles positifs
non triviaux dans la dynamique de Collatz $T(n) = n/2$ ($n$ pair),
$T(n) = (3n+1)/2$ ($n$ impair). En revisitant l'équation de
Steiner~(1977) sous l'angle de la théorie de l'information, nous
identifions un déficit entropique universel
\[
  \gamma = 1 - h\!\left(\frac{1}{\log_2 3}\right) \approx 0{,}0500
\]
où $h$ désigne l'entropie binaire de Shannon. Ce déficit exprime
le fait que le taux de croissance du nombre de compositions admissibles
est strictement inférieur au taux de croissance du module cristallin
$d = 2^S - 3^k$. Il en résulte un
\textbf{Théorème de Non-Surjectivité} (inconditionnel): pour tout
cycle candidat de longueur $k \geq 18$ avec $d > 0$, l'application
d'évaluation modulaire $\Ev_d$ ne peut pas être surjective. Conjugué
au résultat computationnel de Simons et de Weger~\cite{SdW2005},
nous obtenons un \textbf{Théorème de Jonction} couvrant tout $k \geq 2$.
La question résiduelle est formulée comme une \textbf{Hypothèse
d'Équirépartition Exponentielle}~(H), abordée via l'analyse de sommes
de caractères additifs et les bornes de type Weil.
\end{abstract}

\maketitle

% ============================================================
\section{Introduction}
% ============================================================

\subsection{Le problème des cycles}

La conjecture de Collatz (1937) affirme que l'itération
\[
  T(n) = \begin{cases} n/2 & \text{si } n \text{ est pair,} \\
  (3n+1)/2 & \text{si } n \text{ est impair,} \end{cases}
\]
ramène tout entier positif à~$1$. Parmi les stratégies de résolution,
le \textbf{sous-problème des cycles} occupe une place centrale: il
s'agit de démontrer qu'il n'existe aucun cycle positif non trivial.

Un tel cycle comporte $k$ étapes impaires et $S$ étapes paires,
avec $S/k$ proche de $\log_2 3 \approx 1{,}585$.

\subsection{L'équation de Steiner}

Steiner~\cite{Steiner1977} a montré que tout cycle positif de
longueur~$k$ satisfait l'identité arithmétique:
\begin{equation}\label{eq:steiner}
  n_0 \cdot (2^S - 3^k) = \corrSum(A_0, \ldots, A_{k-1})
\end{equation}
où $d = 2^S - 3^k$ est le \emph{module cristallin},
$\corrSum = \sum_{i=0}^{k-1} 3^{k-1-i} \cdot 2^{A_i}$ est la
\emph{somme correctrice}, et $(A_0, \ldots, A_{k-1}) \in \Comp(S,k)$
est une composition admissible avec $A_0 = 0$.

\subsection{Approches antérieures}

\textbf{(i) Bornes computationnelles.} Steiner~\cite{Steiner1977},
puis Simons et de Weger~\cite{SdW2005}, ont utilisé la théorie de
Baker des formes linéaires en logarithmes pour démontrer qu'il
n'existe aucun cycle positif non trivial de longueur $k \leq 68$.

\textbf{(ii) Vérifications de convergence.} Barina~\cite{Barina2021}
a montré que tout entier $n < 2^{68}$ converge vers~$1$.

\textbf{(iii) Approches probabilistes.} Tao~\cite{Tao2022} a
démontré que presque toutes les orbites atteignent des valeurs
arbitrairement petites.

\textbf{(iv) Bornes combinatoires.} Eliahou~\cite{Eliahou1993}
a obtenu des bornes inférieures sur la longueur des cycles non
triviaux. Notre approche se distingue par l'identification de la
constante universelle $\gamma \approx 0{,}05004$, le seuil explicite
$K_0 = 18$, et le cadre information-théorique reliant le problème
à la capacité de canal de Shannon.

\subsection{Notre contribution}

Nous étudions la \textbf{cardinalité de l'image} de l'application
d'évaluation modulaire
\[
  \Ev_d : \Comp(S,k) \to \Z/d\Z, \quad A \mapsto \corrSum(A) \bmod d.
\]
Nous proposons la première formalisation explicite de la
non-surjectivité modulaire de $\Ev_d$ fondée sur le déficit
entropique: la constante $\gamma \approx 0{,}05004$ interdit à
$\Ev_d$ d'être surjective dès que $k \geq 18$. Ce résultat ne
repose sur aucune hypothèse non démontrée.

% ============================================================
\section{Préliminaires et notations}
% ============================================================

\subsection{Compositions admissibles}

\begin{definition}
Pour des entiers $S > k \geq 1$, l'ensemble des
\textbf{compositions admissibles} est:
\[
  \Comp(S,k) = \{(A_0, \ldots, A_{k-1}) \in \Z^k :
    0 = A_0 < A_1 < \cdots < A_{k-1} \leq S-1\}.
\]
Son cardinal est $|\Comp(S,k)| = \binom{S-1}{k-1}$.
\end{definition}

\subsection{Convergents de $\log_2 3$}

Le développement en fraction continue de $\log_2 3$ est
$[1; 1, 1, 2, 2, 3, 1, 5, 2, 23, \ldots]$. Les convergents
$p_n/q_n$ fournissent les meilleures approximations rationnelles.
Les convergents d'index impair donnent $d > 0$ (cycles positifs):

\begin{center}
\begin{tabular}{ccccc}
\toprule
$n$ & $p_n$ & $q_n$ & $d_n = 2^{p_n} - 3^{q_n}$ & signe \\
\midrule
1 & 2 & 1 & 1 & $+$ \\
3 & 8 & 5 & 13 & $+$ \\
5 & 65 & 41 & $\approx 4{,}20 \times 10^{17}$ & $+$ \\
7 & 485 & 306 & $\approx 2^{475}$ & $+$ \\
9 & 24727 & 15601 & $\approx 2^{24711}$ & $+$ \\
\bottomrule
\end{tabular}
\end{center}

\subsection{Entropie binaire de Shannon}

Pour $p \in (0,1)$, l'entropie binaire est
$h(p) = -p \log_2 p - (1-p)\log_2(1-p)$.
L'approximation de Stirling donne
$\log_2 \binom{n}{m} \approx n \cdot h(m/n) + O(\log n)$.

% ============================================================
\section{Le Gap Entropie-Module}
% ============================================================

\subsection{Taux entropique des compositions}

En posant $\alpha = k/S \to 1/\log_2 3 \approx 0{,}6309$:
\[
  \log_2 |\Comp(S,k)| \approx S \cdot h(\alpha).
\]

\subsection{La constante $\gamma$}

\begin{definition}
Le \textbf{gap entropie-module} est:
\[
  \gamma = 1 - h\!\left(\frac{\ln 2}{\ln 3}\right)
  = 0{,}05004447281167\ldots
\]
\end{definition}

\textbf{Calcul.} Posons $\alpha = \ln 2/\ln 3 = 0{,}63093\ldots$
Alors $h(\alpha) = 0{,}41922 + 0{,}53074 = 0{,}94996$, d'où
$\gamma = 1 - 0{,}94996 = 0{,}05004$.

\subsection{Interprétation}

La constante $\gamma$ mesure le déficit informationnel par bit:
\[
  \log_2(C/d) \approx -\gamma \cdot S + \log_2(a_{n+1}) + O(\log S).
\]
Le terme $-\gamma S$ (poids entropique) pousse $C/d$ vers~$0$
exponentiellement.

% ============================================================
\section{Le Théorème de Non-Surjectivité}
% ============================================================

\subsection{Énoncé}

\begin{theorem}[Non-surjectivité cristalline]\label{thm:nonsurj}
Soit $k \geq 18$ un entier et $S = \lceil k \cdot \log_2 3 \rceil$.
Si $d = 2^S - 3^k > 0$, alors
\[
  \binom{S-1}{k-1} < d.
\]
En conséquence, l'application $\Ev_d : \Comp(S,k) \to \Z/d\Z$
n'est pas surjective.
\end{theorem}

\subsection{Démonstration}

\textbf{Étape 1 (Borne asymptotique).} Par Stirling:
$\log_2 C \leq S(1-\gamma) + O(\log S)$.
Pour les convergents, $\log_2 d \geq S - 1$, d'où
$\log_2(C/d) \leq -\gamma S + O(\log S)$.

\textbf{Étape 2 (Non-convergents).} Pour $k \neq q_n$,
$d(k) \geq d(q_n)$ tandis que le taux entropique reste
$\approx 1 - \gamma$.

\textbf{Étape 3 (Vérification numérique).} Pour
$k \in [2, 500]$, calcul exact: $C/d < 1$ pour tout
$k \geq 18$ avec $d > 0$. Exceptions: $k \in \{3, 5, 17\}$
(toutes $< 68$).

\begin{center}
\begin{tabular}{ccccc}
\toprule
$k$ & $S$ & $C(S{-}1, k{-}1)$ & $d$ & $C/d$ \\
\midrule
3 & 5 & 6 & 5 & 1{,}20 \\
5 & 8 & 35 & 13 & 2{,}69 \\
17 & 27 & 5\,311\,735 & 5\,077\,565 & 1{,}05 \\
\bottomrule
\end{tabular}
\end{center}

\textbf{Étape 4 (Borne de Baker pour $k \geq 500$).}
Par les résultats effectifs de Laurent, Mignotte et
Nesterenko~\cite{LMN1995}:
\[
  \log_2(C/d) \leq -k \cdot \gamma \cdot \log_2 3 + C_B'(\log_2 k)^2 + O(\log k).
\]
L'inégalité structurelle décisive $\gamma > 0$ (i.e.\
$h(\ln 2/\ln 3) < 1$) garantit $\log_2(C/d) \to -\infty$.
\qed

\subsection{Le seuil $K_0 = 18$}

Le convergent frontière est $q_5 = 41$, pour lequel
$C/d \approx 0{,}596$ --- le premier convergent d'index
impair où le déficit entropique l'emporte.

% ============================================================
\section{Le Théorème de Jonction}
% ============================================================

\begin{theorem}[Jonction]\label{thm:junction}
Pour tout entier $k \geq 2$, au moins l'une des deux obstructions
suivantes s'applique:
\begin{enumerate}[label=(\Alph*)]
  \item \textbf{Obstruction computationnelle}: si $k \leq 68$, aucun
    cycle positif non trivial n'existe~\cite{SdW2005}.
  \item \textbf{Obstruction entropique}: si $k \geq 18$ et
    $d = 2^{\lceil k \cdot \log_2 3 \rceil} - 3^k > 0$, alors $\Ev_d$
    n'est pas surjective.
\end{enumerate}
L'intersection $[18, 67]$ assure la couverture complète.
\end{theorem}

\begin{proof}
(A) est le résultat de Simons--de~Weger. (B) est le
Théorème~\ref{thm:nonsurj}. L'intersection est immédiate:
$18 \leq 67 < 68$. \qedhere
\end{proof}

\subsection{Architecture des trois régimes}

\begin{center}
\begin{tabular}{lcccc}
\toprule
Convergent & $k$ & $S$ & $\log_2(C/d)$ & Couverture \\
\midrule
$q_3$ & 5 & 8 & $+1{,}43$ & Simons--de~Weger \\
$q_5$ & 41 & 65 & $-0{,}75$ & SdW $+$ Non-surj.\ \\
$q_7$ & 306 & 485 & $-19{,}7$ & Non-surjectivité \\
$q_9$ & 15601 & 24727 & $-1230$ & Non-surjectivité \\
\bottomrule
\end{tabular}
\end{center}

% ============================================================
\section{L'Hypothèse d'Équirépartition Exponentielle}
% ============================================================

\subsection{Le résidu $0$}

Les Théorèmes~\ref{thm:nonsurj} et~\ref{thm:junction} établissent
que $\Ev_d$ omet des résidus. L'existence d'un cycle requiert
spécifiquement $0 \in \operatorname{Im}(\Ev_d)$, c'est-à-dire
$d \mid \corrSum(A)$ pour une composition~$A$.

\subsection{L'Hypothèse (H)}

\begin{hypothesis}[Équirépartition exponentielle]\label{hyp:H}
Pour tout premier $p \mid d$ avec $\operatorname{ord}_p(2)$
suffisamment grand, les sommes de caractères satisfont une
annulation de type Weil: pour tout caractère non trivial~$\chi$
de $\F_p^{\times}$,
\[
  \left| \sum_{A \in \Comp(S,k)} \chi(\corrSum(A)) \right|
  \leq \binom{S-1}{k-1} \cdot p^{-1/2+\varepsilon}
\]
pour tout $\varepsilon > 0$ et $k$ suffisamment grand.
\end{hypothesis}

\subsection{Conséquence de (H)}

Sous l'Hypothèse~(H), par orthogonalité des caractères:
\[
  |\{A : \corrSum(A) \equiv 0 \bmod p\}|
  = \frac{C}{p}\bigl(1 + O(p^{-1/2+\varepsilon})\bigr).
\]
Conjuguée au Théorème de Jonction, (H) implique l'inexistence
complète des cycles positifs non triviaux.

\subsection{Éléments en faveur de (H)}

\textbf{(i)} Vérification numérique pour $q_5$: $\Ev_p$
surjective pour chaque $p \mid d_5$.
\textbf{(ii)} Biais par caractère mod~29:
$(25/28)^{40} \approx 0{,}01$.
\textbf{(iii)} Quasi-injectivité de Horner pour
$\operatorname{ord}_p(2) \gg 1$.
\textbf{(iv)} Cohérence avec Tao~\cite{Tao2022}.

% ============================================================
\section{Obstruction structurelle et vérification formelle}
% ============================================================

\subsection{Le moule multidimensionnel (Phase~14)}

\begin{lemma}[Valuation 2-adique]\label{lem:v2}
Pour toute composition $A \in \Comp(S,k)$ avec $A_0 = 0$,
$\corrSum(A)$ est impair: $v_2(\corrSum(A)) = 0$.
\end{lemma}

\begin{proof}
Le terme $i = 0$ vaut $3^{k-1}$ (impair). Pour $i \geq 1$,
$A_i \geq 1$ donc chaque terme est pair. La somme est impaire.
\end{proof}

\begin{lemma}[Empreinte 2-adique]\label{lem:fingerprint}
Pour toute composition $A = (0, A_1, \ldots, A_{k-1})$:
\[
  \corrSum(A) \equiv 3^{k-1} \pmod{2^{A_1}}.
\]
\end{lemma}

\begin{proof}
Seul le terme $i = 0$ contribue aux bits de position
$0, \ldots, A_1 - 1$. Les termes $i \geq 1$ s'annulent
modulo~$2^{A_1}$.
\end{proof}

\begin{theorem}[Borne du moule multidimensionnel]\label{thm:mold}
Pour $k \geq 18$, la fraction des compositions atteignant un
résidu donné modulo~$d$ est bornée par $1/d \to 0$
exponentiellement.
\end{theorem}

\subsection{La tension inter-dimensionnelle (Phase~15)}

\begin{definition}[Classification des premiers cristallins]
Soit $p$ un premier divisant $d$ et $\omega = \operatorname{ord}_p(2)$.
\begin{itemize}
  \item \textbf{Type~I}: $3 \in \langle 2 \rangle \bmod p$.
  \item \textbf{Type~II}: $3 \notin \langle 2 \rangle \bmod p$.
\end{itemize}
\end{definition}

Le premier $p = 929$ ($929 \mid d_7$) est le premier Type~II:
$\operatorname{ord}_{929}(2) = 464 = \frac{929-1}{2}$ et
$\left(\frac{3}{929}\right) = -1$.

\begin{theorem}[Exclusion du zéro pour $q_3$]\label{thm:zero-q3}
Pour $k=5$, $S=8$, $d=13$: $0 \notin \operatorname{Im}(\Ev_{13})$.
Vérifié exhaustivement sur les 35 compositions de $\Comp(8,5)$.
\end{theorem}

\begin{proposition}[Décomposition additive]\label{prop:decomp}
Pour toute composition~$A$:
$\corrSum(A) = 3^{k-1} + V(A)$ où $V(A)$ est toujours pair.
\end{proposition}

\textbf{Loi d'incompatibilité universelle.} L'irrationalité de
$\log_2 3$ se manifeste à trois niveaux:
\begin{enumerate}
  \item \textbf{Archimédien}: $2^S \neq 3^k$ pour
    $(S,k) \neq (0,0)$.
  \item \textbf{Entropique}: $h(1/\log_2 3) < 1 \Rightarrow
    \gamma > 0 \Rightarrow C < d$ pour $k \geq 18$.
  \item \textbf{$p$-adique}: Aux premiers Type~II, la rigidité
    de coset interdit à~$0$ d'être atteint.
\end{enumerate}

\subsection{Obstruction analytique par sommes de caractères (Phase~16)}

Pour un premier $p \mid d$, la condition $\corrSum(A) \equiv 0 \pmod{p}$
se reformule via les caractères additifs de $\Z/p\Z$:
\begin{equation}\label{eq:N0}
  N_0(p) = \frac{C}{p} + \frac{1}{p}\sum_{t=1}^{p-1} T(t)
\end{equation}
où $T(t) = \sum_{A \in \Comp(S,k)} e(t \cdot \corrSum(A)/p)$
est la somme exponentielle associée.

\begin{theorem}[Coût de Parseval]\label{thm:parseval}
Si $N_0(p) \geq 1$ (existence d'un cycle), alors:
\[
  \sum_{t=1}^{p-1} |T(t)|^2 \geq \frac{(p-C)^2}{p-1}.
\]
Dans le régime cristallin ($C \ll p$), cette borne est $\geq p$.
\end{theorem}

\begin{proof}
Par Parseval: $\sum_{t} |T(t)|^2 = p \sum_r N_r^2$.
Si $N_0 \geq 1$, par Cauchy--Schwarz:
$\sum_r N_r^2 \geq 1 + (C-1)^2/(p-1)$.
En soustrayant $|T(0)|^2 = C^2$:
$\sum_{t \neq 0} |T(t)|^2 \geq (p-C)^2/(p-1)$.
\end{proof}

\begin{theorem}[Exclusion conditionnelle]\label{thm:cond}
Sous des bornes uniformes $|T(t)| \leq C \cdot \omega^{-\delta}$
($\omega = \operatorname{ord}_p(2)$, $\delta > 0$), on a
$N_0(p) = 0$ dès que $C(1/p + \omega^{-\delta}) < 1$.
\end{theorem}

\begin{proposition}[Stratégie CRT]\label{prop:crt}
Il suffit de trouver un unique premier cristallin~$p \mid d$
pour lequel $N_0(p) = 0$ afin de conclure à l'inexistence de
tout cycle de longueur~$k$.
\end{proposition}

\subsection{Le Programme Merle: assemblage du théorème final (Phase~18)}

La Phase~18 assemble les résultats des Phases 14--17 en un
\textbf{unique cadre de preuve par l'absurde} à quatre organes:
\begin{enumerate}
  \item \textbf{Le C\oe ur} (Moteur Entropique): $\gamma > 0 \Rightarrow C < d$;
  \item \textbf{Les Jambes} (Fondation $p$-adique): Newton, Horner, Hensel, cosets;
  \item \textbf{Les Bras} (Étau CRT): réduction à un unique $p \mid d$;
  \item \textbf{La Tête} (Cerveau Analytique): borne sur $T(t)$.
\end{enumerate}
Les trois premiers organes sont \textbf{prouvés}. Le quatrième est
\textbf{conditionnel} sur la Conjecture~M.

\begin{hypothesis}[Conjecture M --- Programme Merle]\label{hyp:M}
Il existe des constantes computables $K_1$ et $\delta > 0$ telles
que pour tout $k \geq K_1$, tout premier $p \mid d$ et tout
$t \in \{1, \ldots, p-1\}$:
\[
  |T(t)| \leq \binom{S-1}{k-1} \cdot k^{-\delta}.
\]
\end{hypothesis}

\begin{theorem}[Assemblage]\label{thm:assembly}
Sous la Conjecture~M, pour tout $k \geq 2$, il n'existe aucun cycle
positif non trivial de longueur~$k$ dans la dynamique de Collatz.
\end{theorem}

Le Programme Merle réduit la conjecture de Collatz (pour les cycles
positifs) à un unique énoncé analytique sur les sommes exponentielles
lacunaires en caractéristique finie.

\subsection{Le Radar de Mellin: obstruction multiplicative (Phase~19)}

La Phase~19 introduit le \textbf{Radar de Mellin}: une reformulation
multiplicative des sommes exponentielles via les caractères de~$\F_p^*$.

\begin{theorem}[Pont Mellin-Fourier]\label{thm:mellin-bridge}
Pour tout $t \in \{1, \ldots, p-1\}$:
\[
  T(t) = N_0 + \frac{1}{p-1}\sum_{\chi} \tau(\bar\chi)\,\chi(t)\,M(\chi)
\]
où $M(\chi) = \sum_{\substack{A \in \Comp(S,k)\\ \corrSum(A) \not\equiv 0}} \chi(\corrSum(A))$
est la \emph{somme de Mellin} et $\tau(\chi)$ la somme de Gauss.
\end{theorem}

\begin{theorem}[Parseval multiplicatif]\label{thm:mellin-parseval}
$\sum_\chi |M(\chi)|^2 = (p-1)\sum_{n \neq 0} S(n)^2$.
\end{theorem}

\begin{hypothesis}[Conjecture $M_{\text{Mellin}}$]\label{hyp:M-mellin}
Il existe $\varepsilon > 0$ tel que pour tout~$k$ assez grand, tout
$p \mid d$ et tout $\chi$ non trivial:
$|M(\chi)| \leq C^{1-\varepsilon}$.
\end{hypothesis}

Cette reformulation est plus naturelle que la Conjecture~M
(Hypothèse~\ref{hyp:M}): elle opère dans le groupe multiplicatif
$\F_p^*$ et exploite les symétries de la structure lacunaire.
Le pont (Théorème~\ref{thm:mellin-bridge}) est une \textbf{identité
exacte}; il n'affaiblit aucun résultat existant mais enrichit
l'arsenal analytique d'une vision multiplicative.

\subsection{Vérification formelle en Lean~4}

Les résultats computationnels ont été formalisés en Lean~4 (v4.15.0)
avec \textbf{0~sorry} et \textbf{0~axiom}. Le fichier
\texttt{CollatzVerified/Basic.lean} contient 73 théorèmes prouvés,
dont la non-surjectivité pour $k = 18, \ldots, 25$, l'exclusion du
zéro pour $q_3$, la classification de $p = 929$ comme Type~II,
la vérification de l'identité de Parseval pour~$q_3$,
les vérifications d'assemblage de la Phase~18,
et le pont Mellin-Fourier avec le Parseval multiplicatif (Phase~19).

% ============================================================
\section{Conclusion}
% ============================================================

Nous avons démontré que le problème des cycles positifs de Collatz
est gouverné par un déficit entropique fondamental
$\gamma = 0{,}05004\ldots$, qui rend $\Ev_d$ non surjective pour
tout $k \geq 18$. Conjugué à Simons--de~Weger ($k \leq 68$), cela
produit un Théorème de Jonction couvrant tout $k \geq 2$.

L'analyse structurelle des Phases 14--15 identifie une loi
d'incompatibilité universelle entre les bases 2 et~3 (archimédienne,
entropique, $p$-adique). La Phase~16 traduit l'Hypothèse~(H) dans
le langage des sommes de caractères, établissant inconditionnellement
le coût de Parseval (Théorème~\ref{thm:parseval}) et réduisant le
problème à l'exclusion du zéro pour un unique premier cristallin
via le CRT (Proposition~\ref{prop:crt}).

L'ensemble des résultats computationnels a été formalisé en
\textbf{Lean~4 avec 0~sorry et 0~axiom} (73 théorèmes).

La Phase~18 (Programme Merle) assemble ces résultats en un unique
cadre de preuve par l'absurde à quatre organes et réduit le problème
à la \textbf{Conjecture~M} (Hypothèse~\ref{hyp:M}): une borne
lacunaire de Fourier $|T(t)| \leq C \cdot k^{-\delta}$.
La Phase~19 (Radar de Mellin) reformule cette conjecture dans le
langage multiplicatif: la \textbf{Conjecture~$M_{\text{Mellin}}$}
(Hypothèse~\ref{hyp:M-mellin}) est plus naturelle et opère dans
$\F_p^*$. Le pont Mellin-Fourier (Théorème~\ref{thm:mellin-bridge})
relie les visions additive et multiplicative par une identité exacte.

\textbf{Vérification numérique de la Condition~(Q).}
Une investigation systématique (méthodologie GPS, 6~phases,
${\sim}30$~expériences computationnelles) a vérifié la condition
quantitative $|\sum_{t=1}^{p-1} T(t)| \leq 0{,}041 \cdot C$ pour
tout $k \in [18, 28]$ et tous les premiers $p \mid d(k)$ testés
(25~cas, 0~échecs). Le pire cas est $k = 22$, $p = 7$, avec un
ratio de~$0{,}013$ (marge $3{,}2\times$). La distribution $N_r(p)$
est approximativement constante sur les orbites de multiplication
par~$2$ dans~$\F_p$ (déviation maximale $2{,}2\%$), et un taux de
décroissance $\sim k^{-6{,}3}$ est observé pour $p = 7$ ($k = 22$
à~$38$).

\medskip\noindent
\textbf{Filet à trois mailles.}
Une structure de preuve à trois niveaux couvre tous les facteurs
premiers de $2^m - 1$ pour $m = 1, \ldots, 100$ ($168$~premiers,
$0$~échecs) :
\begin{enumerate}[label=(\roman*)]
  \item \emph{Transport affine} : couvre tous les $p \leq 97$
    inconditionnellement, via l'identité de commutateur
    $[T_2, T_1] = \tau_{-1}$ ;
  \item \emph{Contraction par convolution} : pour les premiers où
    $(p-1)\,\rho_p^{17} < 0{,}041$, la borne à $k = 18$ suffit
    ($72$~premiers) ;
  \item \emph{Poissons fantômes} : pour les $72$~premiers restants
    (y compris les Mersenne jusqu'à $M_{89}$), la convolution
    échoue à $k = 18$ mais aucune divisibilité $p \mid d(k)$
    n'apparaît dans la zone danger $[18, k_{\min}(p))$.
\end{enumerate}
La vérification directe pour tous les Mersenne $M_q$ connus avec
$q \leq 127$ confirme la propriété fantôme (jusqu'à $3738$ valeurs
de~$k$ vérifiées pour $M_{127}$). Deux barrières indépendantes
protègent les plus grands Mersenne : une \emph{barrière de taille}
($k \geq q \log_3 2 \approx 0{,}63\,q$, rigoureuse) et une
\emph{barrière de densité} (nombre attendu d'occurrences
$E \leq C\,q^3/2^q \to 0$ super-exponentiellement, heuristique).
Pour $q \geq 61$, le nombre attendu de divisibilités dangereuses
est inférieur à~$10^{-14}$.

Une stratégie hybride combinant le filet à trois mailles
($k \leq 40$) et la décroissance asymptotique ($k \geq 41$) apparaît
réalisable (estimation $4/5$) pour établir la Condition~(Q) pour
tout $k \geq 18$.

\emph{Limitation.} Le présent travail ne traite que des cycles
positifs ($d > 0$). L'analyse des cycles négatifs fera l'objet
d'un travail ultérieur. Mentionnons que Böhm--Sontacchi~\cite{BS1978}
et Steiner~\cite{Steiner1977} ont traité les deux signes.

% ============================================================
% Bibliography
% ============================================================

\begin{thebibliography}{13}
\bibitem{Crandall1978} R.~E.~Crandall, On the $3x+1$ problem,
  \emph{Math.\ Comp.}\ \textbf{32} (1978), 1281--1292.
\bibitem{Eliahou1993} S.~Eliahou, The $3x+1$ problem: new lower
  bounds on nontrivial cycle lengths, \emph{Discrete Math.}\
  \textbf{118} (1993), 45--56.
\bibitem{Lagarias1985} J.~C.~Lagarias, The $3x+1$ problem and its
  generalizations, \emph{Amer.\ Math.\ Monthly}\ \textbf{92} (1985),
  3--23.
\bibitem{LMN1995} M.~Laurent, M.~Mignotte, Y.~Nesterenko, Formes
  linéaires en deux logarithmes et déterminants d'interpolation,
  \emph{J.\ Number Theory}\ \textbf{55} (1995), 285--321.
\bibitem{SdW2005} D.~Simons, B.~de~Weger, Theoretical and
  computational bounds for $m$-cycles of the $3n+1$ problem,
  \emph{Acta Arith.}\ \textbf{117} (2005), 51--70.
\bibitem{Steiner1977} R.~P.~Steiner, A theorem on the Syracuse
  problem, \emph{Proc.\ 7th Manitoba Conf.\ Numer.\ Math.}\ (1977),
  553--559.
\bibitem{Tao2022} T.~Tao, Almost all orbits of the Collatz map attain
  almost bounded values, \emph{Forum Math.\ Pi}\ \textbf{10} (2022),
  e12.
\bibitem{Barina2021} T.~Barina, Convergence verification of the
  Collatz problem, \emph{J.\ Supercomput.}\ \textbf{77} (2021),
  2681--2688.
\bibitem{Wirsching1998} G.~J.~Wirsching, \emph{The Dynamical System
  Generated by the $3n+1$ Function}, Lecture Notes in Math.\
  \textbf{1681}, Springer, 1998.
\bibitem{BS1978} C.~Böhm, G.~Sontacchi, On the existence of cycles
  of given length in integer sequences, \emph{Atti Accad.\ Naz.\
  Lincei}\ \textbf{64} (1978), 260--264.
\bibitem{Lagarias2010} J.~C.~Lagarias (ed.), \emph{The Ultimate
  Challenge: The $3x+1$ Problem}, Amer.\ Math.\ Soc., 2010.
\bibitem{KM2005} A.~V.~Kontorovich, S.~J.~Miller, Benford's law,
  values of $L$-functions and the $3x+1$ problem, \emph{Acta Arith.}\
  \textbf{120} (2005), 269--297.
\bibitem{Rozier2015} O.~Rozier, The $3x+1$ problem: a lower bound
  hypothesis, \emph{preprint}, 2015.
\bibitem{NAM2022} W.~Ngom, D.~Alpay, M.~Mboup, Discrete-time
  signals and systems on the Poincaré disk: scale shift and the
  Mellin transform, \emph{Signal Processing}\ \textbf{199} (2022),
  108613.
\bibitem{Kuznetsov2007} A.~Kuznetsov, Expansion of the Riemann
  $\Xi$ function in Meixner-Pollaczek polynomials, \emph{Canad.\
  Math.\ Bull.}\ \textbf{51} (2008), 185--193.
\end{thebibliography}

\end{document}
